
\section{Event-related paradigms}


MEG (magnetoencephalography) and EEG (electroencephalography) are non-invasive tools for studying large-scale network dynamics in the human brain. These methods allow researchers to explore meso- and macro-scale neural activity, linking local oscillatory dynamics to global network behavior. By leveraging their temporal precision, MEG and EEG provide critical insights into the coordination of neuronal populations in cognition and perception \cite{gross2019meg}.

\subsubsection*{Electrophysiological Signals and Source Modeling}

MEG and EEG signals arise from the electrical activity of neuronal populations. While MEG captures magnetic fields generated by coherent primary currents in cortical pyramidal cells, EEG records the electric fields induced by extracellular return currents. The physical mechanisms of signal generation differ, making MEG and EEG complementary tools for capturing distinct neural dynamics \cite{palva2012functional}.

Source modeling is essential to localize brain activity accurately. This process includes:
\begin{itemize}
    \item Building a \textit{head model} using anatomical data (e.g., MRI) to simulate current flow through brain tissues.
    \item Defining a \textit{source model} that estimates the distribution of current generators.
    \item Applying forward and inverse operators to link sensor-space data to brain sources \cite{gross2019meg}.
\end{itemize}

Source modeling addresses challenges like signal mixing and volume conduction, which can obscure the true origin of neural activity.

\subsubsection*{Preprocessing MEG and EEG Data}

Preprocessing prepares MEG and EEG data for analysis by attenuating non-neuronal signals. Key steps include:
\begin{itemize}
    \item \textbf{Artifact Rejection:} Removing noise from eye blinks, muscle activity, and heartbeat using independent component analysis (ICA).
    \item \textbf{Filtering:} Employing band-stop (notch) filters to eliminate line noise (e.g., 50 Hz in Europe, 60 Hz in the USA).
    \item \textbf{Cortical Parcellation:} Collapsing fine-grained source models (5,000–8,000 sources per hemisphere) into cortical parcels (100–500 regions) based on anatomy or functionality (e.g., the Schaefer atlas) \cite{lobier2017functional}.
\end{itemize}

This pipeline facilitates the transition from sensor-space data to meaningful interpretations of large-scale network activity.

\subsubsection*{Handling Source Leakage and Spurious Couplings}

Source leakage, or residual linear mixing, occurs when signals from multiple brain areas overlap at a single sensor. This mixing can inflate measures like the Phase Locking Value (PLV), leading to spurious couplings. Advanced metrics address these issues:
\begin{itemize}
    \item \textbf{Imaginary PLV (iPLV):} Removes zero-lag components that arise from source leakage, isolating genuine phase relationships.
    \item \textbf{Orthogonalized Correlation Coefficient (oCC):} Reduces artificial amplitude correlations caused by linear mixing \cite{hipp2012functional}.
\end{itemize}

Despite these improvements, "ghost" couplings—false positives caused by residual leakage—remain a significant challenge. Techniques like hyperedge bundling cluster interactions likely to stem from the same underlying source, improving accuracy in connectivity analyses \cite{wang2018hyperedge}.

\subsubsection*{Mapping Functional Networks}

By analyzing phase and amplitude synchronization, MEG and EEG enable the mapping of functional brain networks. For example:
\begin{itemize}
    \item \textbf{Inter-Areal Synchronization:} Measures like iPLV reveal how different brain regions coordinate during tasks.
    \item \textbf{Task-Specific Networks:} For instance, during attention tasks, alpha-band (8–12 Hz) oscillations modulate visual and frontoparietal network activity, reflecting shifts in attentional focus \cite{lobier2017functional}.
\end{itemize}

Large-scale synchronization networks link regions involved in sensory processing, attention, and cognitive control, offering insights into the dynamics of perception and decision-making.

\subsubsection*{Applications and Implications}

MEG and EEG are powerful tools for investigating neural dynamics across time, frequency, and spatial scales. Their applications include:
\begin{itemize}
    \item \textbf{Clinical Diagnostics:} Identifying network dysfunction in neurological disorders like epilepsy and schizophrenia.
    \item \textbf{Cognitive Neuroscience:} Exploring the neural basis of attention, memory, and perception.
    \item \textbf{Brain-Computer Interfaces (BCIs):} Leveraging real-time EEG signals for communication and control systems.
\end{itemize}

While challenges like source leakage persist, advances in preprocessing, source modeling, and interaction metrics have significantly improved the reliability of MEG and EEG analyses.

\subsubsection*{Conclusion}

This chapter highlighted the use of MEG and EEG for mapping large-scale brain networks. By combining sophisticated preprocessing pipelines with robust synchronization metrics, these tools offer unparalleled insights into the dynamics of functional connectivity. Future work will continue to refine these methods, addressing challenges like ghost couplings and expanding their applications to clinical and experimental contexts.
