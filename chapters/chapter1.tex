\chapter{Introduction}

\rule{\linewidth}{0.4pt} % Horizontal line

\section*{Glossary}

\vspace{3pt}           % Adds 10pt of vertical spa
\begin{itemize}
    \item \textbf{Construct}: a theoretical concept or idea that is inferred to exist based on evidence or reasoning but cannot be directly observed or measured. Examples include intelligence, motivation, and self-esteem \cite{construct_philosophy}.
    \item \textbf{Operationalization}: the process of defining a construct in terms of specific, measurable phenomena or operations that allow it to be observed and quantified. For example, operationalizing "intelligence" through IQ test scores \cite{operationalization}.
    \item \textbf{Mechanism}: a system or process involving the interaction of parts or elements that produces a specific phenomenon or outcome. It explains "how" something occurs by detailing the cause-and-effect relationship \cite{mechanism_philosophy}.
    \item \textbf{Epiphenomenom}:  a secondary effect or byproduct that arises alongside a primary phenomenon but has no direct causal influence on the primary process. For example, the sound of a boiling kettle is an epiphenomenon of the boiling process \cite{epiphenomenon}.
\end{itemize}


\section*{Learning goals}

\begin{itemize}
\item Grasp the significance of oscillations in brain function.
\item Understand the key constructs and their operationalization in neuroscience.
\item Differentiate between mechanism, phenomenon, and epiphenomenon.
\item Appreciate the framework of criticality and its implications for neuronal dynamics.
\end{itemize}

\rule{\linewidth}{0.4pt} % Horizontal line
\vspace{1pt}           % Adds 10pt of vertical spa

%friendly introduction to brain oscillations
From, the heartbeat to the daily circadian cycle, the nature is full of repeating patterns, rhythms.
Already, the earliest electrophysiological recordings revealed a strong rhythmic component in the human brain, the alpha rhythm.
Brain oscillations refer to periodic fluctuations in neuronal activity, observed across various spatial and temporal scales. These rhythmic patterns emerge from the interactions of excitatory and inhibitory neurons within specific microcircuit structures. Oscillations are integral to the timing and synchronization of neuronal firing, facilitating communication within and between brain regions.
Later in chapters 3 and 4, we will discuss how to measure the synchronization between oscillations.

% brain oscillations to brain criticality
To survive, the brain needs to adapt to varying situations. For this reason, the brain has a capability to modulate itself. 
This means that it can operate in various states, which is reflected, among other things, in the synchronization dynamics between brain areas.
The framework of criticality offers methods for analyzing this modulatory capability of the brain.
Critical oscillations combine neuronal rhythms with synchronization dynamics. 
This approach examines how oscillatory activity transitions between states of order and disorder, optimizing the brain’s dynamic range and responsiveness.
In Chapters 5 and 6, we will learn about brain criticality, how it can be measured, and how it manifests in the brain.
Chapter 7 will discuss a modeling approach for studying criticality.

%todo: Introducing chapters 8 and 9

\section{Constructs and Operationalization}

Constructs represent theoretical frameworks that describe unmeasurable concepts. Although non-measurable, the concepts are either inferred or hypothesized to exist. Constructs guide hypotheses, measurements, and analyses in research. 
For example, every physical object has a center of gravity. If you balance the object below its center of gravity, it will not fall. However, you cannot find this any physical representation of the center of gravity inside the object. Center of gravity is a construct.

Operationalization is the process of translating abstract constructs into measurable quantities. 
In the light of our previous example of center of gravity, you may come up with an elaborate procedure to find the balancing points of the physical object, thus operationalizing center of gravity.
For oscillations, this involves quantifying properties such as amplitude and phase using techniques like:
\begin{itemize}
\item \textbf{Power Spectral Analysis:} Decomposes a signal into its frequency components.
\item \textbf{Wavelet Transform:} Provides a time-frequency representation, capturing how oscillatory patterns evolve over time.
\item \textbf{Phase synchronization methods} utilize phase properties to infer how synchronized the underlying signals are.
\end{itemize}

\section{Mechanism, Phenomenon, and Epiphenomenon}

\textbf{Mechanism} is a description of the causal process that governs an observable \textbf{phenomenon}. For example, the interaction between excitatory pyramidal neurons and inhibitory interneurons leads to ionic movement, generating electric currents and associated magnetic fields. When these processes occur collectively within a cortical neural ensemble, they produce signals that can be detected by electrophysiological measurement techniques such as EEG or MEG. These detected signals often exhibit oscillatory patterns and appear localized to the electrodes near the source of the activity. These oscillations are a phenomenon -  in general, phenomena are observable behaviors or events that emerge from their underlying mechanisms. 

\textbf{Epiphenomena}, in contrast, are secondary effects or byproducts that do not directly contribute to the primary function or underlying mechanisms. For instance, in EEG recordings, volume conduction can cause signals from distant brain regions to appear synchronized. This effect occurs because the electrical fields generated by neuronal activity propagate passively through brain tissue, cerebrospinal fluid, the skull, and the scalp. Although these propagated fields affect the observed EEG signal, they do not represent functional interactions or true connectivity between distant brain regions. Instead, volume conduction is a byproduct of the medium through which the signals travel, creating an epiphenomenon that distorts the apparent spatial and temporal structure of the recorded activity. 

By distinguishing mechanisms, phenomena, and epiphenomena, we can better interpret electrophysiological data and separate true neuronal dynamics from misleading artifacts or secondary effects.


\section{Scientific Inquiry in Brain Oscillations}

Identify the neurophysiological construct of interest, such as gamma oscillations, and formulate hypotheses about its role in cognition or behavior. 

\begin{enumerate}
    \item Record oscillatory activity using electrophysiological techniques (e.g., EEG, MEG) and preprocess the data to extract relevant features like amplitude and phase.
    \item Apply statistical and computational models to evaluate the relationships between oscillatory patterns and functional outcomes.
    \item Interpret the results to gain insights into underlying neurobiological processes, informing both theoretical frameworks and practical applications.
\end{enumerate}


\vspace{10pt}           % Adds 10pt of vertical spa
\rule{\linewidth}{0.4pt} % Horizontal line
\vspace{1pt}           % Adds 10pt of vertical spa


\textbf{Questions for Review:}
\begin{enumerate}
\item What are the defining characteristics of neuronal oscillations?
\item Explain the process of operationalizing neuronal oscillations using signal processing techniques.
\end{enumerate}

\textbf{Recommended Reading:}
\begin{itemize}
    \item Construct (Wikipedia) \cite{construct_philosophy}:  \href{https://en.wikipedia.org/wiki/Construct_\%28philosophy\%29}{https://en.wikipedia.org/wiki/Construct\_(philosophy)}
    
    \item Operationalization (Wikipedia) \cite{operationalization}:  \href{https://en.wikipedia.org/wiki/Operationalization}{https://en.wikipedia.org/wiki/Operationalization}

    \item Fries, P. (2015). Rhythms for Cognition: Communication through Coherence. \textit{Neuron}, 88(1), 220–235. \cite{fries2015rhythms}
    

\end{itemize}

