
\section{Criticality - Concepts and Evidence}

Criticality, the state at which complex systems operate near the threshold between order and disorder, provides a foundational framework for understanding neural dynamics. At this state, the brain achieves an optimal balance of stability and adaptability, enabling efficient information processing and dynamic responsiveness to stimuli. This chapter explores the conceptual foundations, modeling approaches, and empirical evidence of criticality in brain systems.

\subsection*{Theoretical Foundations of Criticality}

Criticality is characterized by key phenomena observed during phase transitions:
\begin{itemize}
    \item \textbf{Correlation Length Divergence:} At criticality, interactions propagate across long distances, fostering large-scale coordination.
    \item \textbf{Power-Law Scaling:} Activity follows power-law distributions, indicating scale-invariance in neural dynamics.
    \item \textbf{Dynamic Range Optimization:} Systems maximize their ability to respond effectively across a wide range of input intensities.
\end{itemize}

These principles suggest that the brain leverages criticality to support complex functions, balancing local specialization with global integration.

\subsection*{Modeling Criticality}

Neural criticality is modeled through frameworks such as:
\begin{itemize}
    \item \textbf{Ising Model:} Describing interactions between spins (analogous to neurons), this model captures transitions from ordered to disordered states.
    \item \textbf{Branching Processes:} Representing neuronal activity as a branching network, criticality occurs when each active neuron excites one other neuron on average ($\sigma = 1$).
\end{itemize}

These models provide insights into how critical dynamics emerge from local interactions and propagate across networks.

\subsection*{Empirical Evidence for Criticality}

Experimental studies reveal hallmarks of criticality in neural systems:
\begin{itemize}
    \item \textbf{Neuronal Avalanches:} Discrete bursts of cortical activity exhibit power-law scaling, reflecting a balance between stability and variability.
    \item \textbf{Long-Range Temporal Correlations (LRTCs):} Neural signals demonstrate scale-free dynamics, indicative of self-organized criticality.
    \item \textbf{1/f Scaling:} The characteristic scaling of power spectra highlights the fractal nature of brain activity.
\end{itemize}

These findings underscore criticality’s role in optimizing neural computation and communication.

\subsection*{Functional Implications}

Operating near criticality confers several advantages:
\begin{itemize}
    \item \textbf{Enhanced Information Transmission:} Criticality maximizes signal propagation and network efficiency.
    \item \textbf{Robustness and Flexibility:} The system remains resilient to perturbations while adapting to novel demands.
    \item \textbf{Cognitive Optimization:} Behavioral studies link critical dynamics to improved attention, memory, and reaction time.
\end{itemize}

These properties highlight criticality’s functional relevance for both healthy cognition and its disruptions in disorders.

\subsection{FROM FIST CHAPTER: The Framework of Criticality}


Criticality describes a system poised at the edge between order and chaos, characterized by maximal responsiveness and adaptability. In the brain, criticality manifests as:
\begin{itemize}
\item \textbf{Neuronal Avalanches:} spontaneous bursts of activity following power-law distributions.
\item \textbf{Long-Range Temporal Correlations (LRTCs):} scale-free fluctuations in neuronal signals.
\item \textbf{Maximized dynamic range:} emerges as a maximal dispersion of observable values, such as inter-areal synchrony levels.
\end{itemize}

Operating near criticality offers several functional advantages, and we will look deeper into them in chapters 5 and 6. 

